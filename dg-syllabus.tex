\documentclass[12pt,article,oneside]{memoir}
\usepackage{comment}
\usepackage{hyperref}
\usepackage{cite}
\usepackage{html}
\author{Jane Sandberg\thanks{You can email me at \htmladdnormallink{sandbej at linnbenton dot edu}{mailto:sandbej@linnbenton.edu}, call me at \htmladdnormallink{(541) 917 4655}	{tel:5419174655}, or stop by my office hours.}}
\title{HD120: Destination Graduation\thanks{3:30-4:20p Thursdays in BC-105}}
\begin{document}
\renewcommand{\labelitemi}{$\triangleright$}
\setcounter{secnumdepth}{0}
\tightlists


\maketitle


\begin{quotation}
something
something

something \sourceatright{Somebody}
\end{quotation}

\begin{htmlonly}
\tableofcontents
\end{htmlonly}

\section{Office hours}

\subsection{Mondays}
10:30am-12pm in WH-143 (my office in the Albany campus library)

\subsection{Thursdays}
4:30-5:30pm in the Benton Center Learning Annex

\section{Required course materials}
\textbf{I do not require any textbooks for this course; please don't buy any.}  All the readings and activities you will need are available as links in this document and Moodle.

\section{Course overview}
Destination Graduation is a ten-hour, one-credit course designed to:
\begin{enumerate}
 \item Help students make a smooth academic and social transition to college life.
 \item Develop students' ability to use tools, information, and resources to be successful at LBCC.
 \item Help students establish a long-term academic advising relationship with a designated academic advisor, most frequently a faculty member.
\end{enumerate}

\section{Learning outcomes}
Upon successful completion of Destination Graduation (DG), you will be able to:
\begin{enumerate}
 \item Identify your responsibilities as a student.
 \item Identify characteristics of successful students, barriers to college success, and strategies for success.
 \item Apply introductory critical thinking skills in a variety of contexts.
 \item Navigate important electronic educational resources. 
 \item Develop a specific Education Plan related to your career/educational goal.
 \item Locate and know how to access support services and educational resources.
 \item Establish relationships with peers, LBCC faculty and staff.
 \item Establish an advising relationship with a designated academic advisor.
\end{enumerate}

\section{Expectations}
\begin{enumerate}
 \item Attend all class sessions
 \item Participate in discussions (if this is an intimidating context for you, talk with me and we can work on a strategy to get you involved)
 \item Complete all assignments
 \item Come to class prepared
\end{enumerate}


\section{Nondiscrimination and non-harassment statement}

Linn-Benton Community College is committed to providing an atmosphere that  encourages individuals to realize their potential. We embrace diversity and inclusion of all persons. The college prohibits unlawful discrimination based on race, color, religion, ethnicity, use of native language, national origin, sex, sexual orientation, marital status, disability, veteran status, or age in any area, activity or operation of the college. In addition, the college complies with related federal, state, and local laws (Civil Rights, Disability \& Rehabilitation Acts, Veterans Acts).

\section{Disability services and emergency planning statement}

Students who may need accommodations or special tools due to a disability or disabilities should contact the Office of Disability Services (ODS) at 541-917-4789 or Red Cedar Hall 101/103 or access information on the LBCC website (\url{http://linnbenton.edu/disability-services}.) Students who have medical information which the instructor should know, or who need special arrangements in an emergency, should notify their instructor and Public Safety (Red Cedar Hall).

\section{Grading}

In order to pass DG a student must:

\begin{enumerate}
 \item Attend a minimum of 70\% of all classes.  Attendance is strongly suggested for all classes.
 \item Complete of a minimum of 70\% of the specified assignments.
 \item Complete an Education Plan and get it signed by their advisor.
 \item Meet with their advisor.
\end{enumerate}
  

\subsection{Extra credit}
Your instructor will provide information about these opportunities in class and on Moodle.


\subsection{Attendance}


\section{Know your rights (and responsibilities)}

LBCC students have rights: the right to free speech, the right to assemble, the right of a free press, etc. LBCC students also have responsibilities to their community: the responsibility to participate and engage in class, the responsibility to advocate for their needs (ask for help), the responsibility to support a respectful teaching and learning environment, the responsibility to treat all persons with respect, the responsibility to be truthful and honest in all work and communications, and the responsibility to follow staff directions, local, state, and federal laws. Rights and responsibilities balance together to create the best learning environment. For example, while you have free speech in the café or courtyard, in class the instructor decides whose turn it is to talk and what the topics for conversation will be. Students are free to believe what they believe, but instructors may require students to learn and recite concepts, principles, or theories for a class even if the student does not believe those concepts. You play a role in creating a positive community at LBCC. 

Please review your rights and responsibilities at this link: \url{http://linnbenton.edu/go/studentrights}.

If you believe a student is violating your rights, ask to be treated with respect. If that does not cure the situation, report to Associate Dean Dr. Lynne Cox, Takena 107.

If you believe a faculty member or LBCC employee is violating your rights, please report to Human Resources, Scott Rolen, CC-108.

\section{Schedule}

All readings and assignments listed are due at the start of the relevant class period.

\subsection{Introduction to the course and important computer systems}

\subsubsection{Due}
No assignments due

\subsubsection{Agenda}
\begin{itemize}
 \item Syllabus
 \item Introductions: names, potential majors, interesting things about you
 \item Email
 \item WebRunner
 \item Moodle
 \item DegreeRunner
 \item Scheduler
\end{itemize}


\subsection{Working in a scholarly community}
\subsubsection{Due}
\begin{itemize}
 \item Assignment: Watch the assigned video \emph{Why OER?}  Then write a response of at least 200 words and submit it via Moodle. 
 \item Viewing: \cite{oer}
 \item Reading: \cite{wellsley}
 \item Supplemental reading: \cite{oer-dh}
\end{itemize}


\subsubsection{Agenda}
\begin{itemize}
 \item Learning Center/Annex
 \item Library
 \item Center for Accessibility Resources
 \item Veteran's Affairs
 \item Diversity Achievement Center
 \item Student Leadership Council
 \item Email and text etiquette
 \item Completing assignments, what to do with instructor feedback
 \item A campus conversation: Open Educational Resources
\end{itemize}



\subsection{Career choices, interdisciplinary thinking}
\subsubsection{Due}
\begin{itemize}
 \item Assignment: 
 \item Reading: \cite{worksheets,ooh}
 \item Supplemental reading: \cite{gen-ed}
\end{itemize}


\subsubsection{Agenda}
\begin{itemize}
 \item Discuss how content of video is relevant to your field
 \item Discuss the possible degrees, programs, and certificates
 \item Spend time exploring Occupational Outlook Handbook
\end{itemize}



\subsection{Financial literacy}
\subsubsection{Due}
\begin{itemize}
 \item Assignment: 
 \item Reading: \cite{finra}
 \item Supplemental reading:
\end{itemize}

\subsubsection{Agenda}
\begin{itemize}
\item 
\end{itemize}



\subsection{Goals and time management}
\subsubsection{Due}
\begin{itemize}
 \item Assignment: 
 \item Reading:
 \item Supplemental reading:
\end{itemize}

\subsubsection{Agenda}
\begin{itemize}
\item 
\end{itemize}

\subsection{Advising}
\subsubsection{Due}
\begin{itemize}
 \item Assignment: 
 \item Reading:
 \item Supplemental reading:
\end{itemize}

\subsubsection{Agenda}
\begin{itemize}
\item 
\end{itemize} 

\subsection{Preparing an educational plan}
\subsubsection{Due}
\begin{itemize}
 \item Assignment: 
 \item Reading:
 \item Supplemental reading:
\end{itemize}

\subsubsection{Agenda}
\begin{itemize}
\item 
\end{itemize}


\subsection{Barriers to college success}
\subsubsection{Due}
\begin{itemize}
 \item Assignment:
 \item Reading: \cite{guillory2008s}
 \item Supplemental reading: \cite{rosenberg}
\end{itemize}

\subsubsection{Agenda}
\begin{itemize}
\item 
\end{itemize}



\subsection{Critical thinking}
\subsubsection{Due}
\begin{itemize}
 \item Assignment: 
 \item Reading:
 \item Supplemental reading:
\end{itemize}

\subsubsection{Agenda}
\begin{itemize}
\item 
\end{itemize}


\subsection{Participating in scholarly conversations}
\subsubsection{Due}
\begin{itemize}
 \item Assignment: A Wikipedia article on something in the local community
 \item Reading:
 \item Supplemental reading:
\end{itemize}

\subsubsection{Agenda}
\begin{itemize}
\item 
\end{itemize}



\newpage
\renewcommand{\bibname}{Reading list}
\bibliography{dg-reading-list}{}
\bibliographystyle{apalike}


\end{document}