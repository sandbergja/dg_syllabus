\documentclass[12pt,article,oneside]{memoir}
\usepackage{comment}
\usepackage{hyperref}
\usepackage{cite}
\usepackage{html}
\author{Jane Sandberg\thanks{You can email me at \htmladdnormallink{sandbej at linnbenton dot edu}{mailto:sandbej@linnbenton.edu}, call me at \htmladdnormallink{(541) 917 4655}	{tel:5419174655}, or stop by my office hours.}}
\title{HD120: Destination Graduation\thanks{3:30-4:20p Thursdays in BC-105}}
\begin{document}
\renewcommand{\labelitemi}{$\triangleright$}
\setcounter{secnumdepth}{0}
\tightlists


\maketitle


\begin{quotation}
Discovery consists of looking at the same thing as everyone else and thinking something different.

\sourceatright{attributed to \htmladdnormallink{Albert Szent-Gy\"{o}rgyi}{ http://search.credoreference.com.ezproxy.libweb.linnbenton.edu:2048/content/topic/szent_gyo\%CC\%88rgyi_albert_1893_1986}}
\end{quotation}

\begin{quotation}
In the classroom, I share as much as possible the need for critical thinkers to engage multiple locations, to address diverse standpoints, to allow us to gather knowledge fully and inclusively.  Sometimes, I tell students, it is like a recipe.  I tell them to imagine we are baking bread that needs flour.  And we have all the other ingredients but no flour.  Suddenly, the flour becomes most important even though it alone will not do.  This is a way to think about [lived] experience in the classroom.

\sourceatright{\htmladdnormallink{bell hooks}{http://search.credoreference.com.ezproxy.libweb.linnbenton.edu:2048/content/topic/hooks_bell_1952}, \emph{Teaching to Transgress}}
\end{quotation}

\begin{htmlonly}
\tableofcontents
\end{htmlonly}
\newpage
\section{Office hours}

\subsection{Mondays}
10:30am-12pm in WH-143 (my office in the Albany campus library)

\subsection{Thursdays}
4:30-5:30pm in the Benton Center Learning Annex

\section{Required course materials}
\textbf{I do not require any textbooks for this course; please don't buy any.}  All the readings and activities you will need are available as links in this document and Moodle.

\section{Course overview}
Destination Graduation is a ten-hour, one-credit course designed to:
\begin{enumerate}
 \item Help students make a smooth academic and social transition to college life.
 \item Develop students' ability to use tools, information, and resources to be successful at LBCC.
 \item Help students establish a long-term academic advising relationship with a designated academic advisor, most frequently a faculty member.
\end{enumerate}

\section{Learning outcomes}
Upon successful completion of Destination Graduation (DG), you will be able to:
\begin{enumerate}
 \item Identify responsibilities and characteristics of successful students and the barriers to college success.
 \item Apply introductory critical thinking skills.
 \item Navigate important electronic educational resources.
 \item Develop a specific Education Plan related to your career/educational goal.
 \item Locate and know how to access support services and educational resources.
 \item Establish relationships with peers, LBCC faculty, staff, and an advising relationship with a designated academic advisor.
\end{enumerate}

\section{Expectations}
\begin{enumerate}
 \item Attend all class sessions
 \item Participate in discussions (if this is an intimidating context for you, talk with me and we can work on a strategy to get you involved)
 \item Complete all assignments
 \item Come to class prepared
\end{enumerate}


\section{Grading}

In order to pass DG a student must:

\begin{enumerate}
 \item Attend a minimum of 70\% of all classes.  Attendance is strongly suggested for all classes.
 \item Complete of a minimum of 70\% of the specified assignments.
 \item Complete an Education Plan and get it signed by their advisor.
 \item Meet with their advisor.
\end{enumerate}

  
\subsection{Extra credit}
Your instructor will provide information about these opportunities in class and on Moodle.

\subsection{File formats}
You will be creating most of your assignments in some text editor and submitting them via Moodle.  I am very flexible about which file format you use. In fact, many of the assignments are quite short, so I encourage you to use them to experiment with tools that you are unfamiliar with.  Here are some ideas:
\begin{itemize}
 \item Typical word processor formats, such as those used by Microsoft Word and OpenOffice/LibreOffice
 \item PDF formats
 \item HTML, \LaTeX, Docbook, or other typical markup languages
 \item Submit a link to a Google doc, YouTube video, blog post, or other online document you've created that meets the stated criteria 
\end{itemize}



However, there are a few formats that are a total pain to open -- particularly documents created using Apple's \emph{Pages} product.  Please don't submit Pages documents. If you would like to submit something in an offbeat file format not listed here, check with me first to make sure that I can access it with my tools.



\section{Unique features of this DG section}
Like your colleagues in other sections of DG, you will be learning about campus resources and planning for a successful education here at LBCC.  However, a key part of your work at LBCC will be participation in scholarly conversations.  To prepare you for this responsibility, we will spend time thinking about research methodologies, documentation of sources, claims and evidence, novel concepts, and knowledge gaps.  During the last part of this term, we will use these discussions to inform lasting contributions to a frequently-used knowledge base, Wikipedia.



\section{Nondiscrimination and non-harassment statement}

Linn-Benton Community College is committed to providing an atmosphere that  encourages individuals to realize their potential. We embrace diversity and inclusion of all persons. The college prohibits unlawful discrimination based on race, color, religion, ethnicity, use of native language, national origin, sex, sexual orientation, marital status, disability, veteran status, or age in any area, activity or operation of the college. In addition, the college complies with related federal, state, and local laws (Civil Rights, Disability \& Rehabilitation Acts, Veterans Acts).

\section{Disability services and emergency planning statement}

Students who may need accommodations or special tools due to a disability or disabilities should contact the Office of Disability Services (ODS) at 541-917-4789 or Red Cedar Hall 101/103 or access information on the LBCC website (\url{http://linnbenton.edu/disability-services}.) Students who have medical information which the instructor should know, or who need special arrangements in an emergency, should notify their instructor and Public Safety (Red Cedar Hall).

\section{Know your rights (and responsibilities)}

LBCC students have rights: the right to free speech, the right to assemble, the right of a free press, etc. LBCC students also have responsibilities to their community: the responsibility to participate and engage in class, the responsibility to advocate for their needs (ask for help), the responsibility to support a respectful teaching and learning environment, the responsibility to treat all persons with respect, the responsibility to be truthful and honest in all work and communications, and the responsibility to follow staff directions, local, state, and federal laws. Rights and responsibilities balance together to create the best learning environment. For example, while you have free speech in the caf\'{e} or courtyard, in class the instructor decides whose turn it is to talk and what the topics for conversation will be. Students are free to believe what they believe, but instructors may require students to learn and recite concepts, principles, or theories for a class even if the student does not believe those concepts. You play a role in creating a positive community at LBCC. 

Please review your rights and responsibilities at this link: \url{http://linnbenton.edu/go/studentrights}.

If you believe a student is violating your rights, ask to be treated with respect. If that does not cure the situation, report to Associate Dean Dr. Lynne Cox, Takena 107.

If you believe a faculty member or LBCC employee is violating your rights, please report to Human Resources, Scott Rolen, CC-108.

\newpage

\section{Schedule}

All readings and assignments listed are due at the start of the relevant class period.

\subsection{Introduction to the course and important computer systems}

\subsubsection{Due}
No assignments due

\subsubsection{Agenda}
\begin{itemize}
 \item Plans for the course
 \item Syllabus
 \item Introductions: names, potential majors, interesting things about you
 \item Email
 \item Moodle
 \item WiFi
 \item Office software options: Office 365, LibreOffice, Google Drive, \LaTeX
\end{itemize}


\subsection{Advising and student support}
\subsubsection{Due}
\begin{itemize}
 \item Assignment: Write at least 200 words about where you see yourself in 5 years.  Include career goals or any other goals that are important to you.
 \item Viewing: \cite{credits}
 \item Supplemental reading: 
\end{itemize}

\subsubsection{Agenda}
\begin{itemize}
\item WebRunner
\item DegreeRunner
\item AdvisorTrac
\item The advising process
\item Center for Accessibility Resources
\item Services for veterans

\end{itemize} 



\subsection{Participating in a scholarly community}
\subsubsection{Due}
\begin{itemize}
 \item Assignment: Watch the assigned video \emph{Why OER?}  Then write a response of at least 200 words and submit it via Moodle. 
 \item Viewing: \cite{oer}
 \item Reading: \cite{wellsley}
 \item Supplemental reading: \cite{oer-dh}
\end{itemize}


\subsubsection{Agenda}
\begin{itemize}
 \item Learning Center/Annex
 \item Library
 \item Diversity Achievement Center
 \item Student Leadership Council
 \item Email and text etiquette
 \item How to make the most of group assignments
 \item A campus conversation: Open Educational Resources
\end{itemize}



\subsection{Preparing for a career}
\subsubsection{Due}
\begin{itemize}
 \item Assignment: Use LBCC or OSU library resources to identify a scholarly article (preferably peer-reviewed) that discusses the skills necessary for a field of study that interests you.  Read the article, and create a document that offers three of these skills, why the author(s) felt they were important, how you can develop them during your time at school, and a complete citation for the article you read.  Upload the document to Moodle.
 \item Reading: \cite{worksheets}
 \item Supplemental reading: \cite{crosswalk,ooh}
\end{itemize}

\subsubsection{Agenda}
\begin{itemize}
 \item Explore Occupational Outlook Handbook
 \item Discuss the degrees, programs, and certificates offered by LBCC
\item Talk about education plans.
\end{itemize}



\subsection{Interdisciplinary thinking, research, time management, the research process}
\subsubsection{Due}
\begin{itemize}
 \item Assignment 1: Prepare a draft educational plan.  Make an appointment with your advisor to develop it further.
 \item Assignment 2: Use the \htmladdnormallink{Research Project Calculator}{https://rpc.elm4you.org/} to create a plan for an assignment in one of your other classes.  Print out a copy for yourself, and take a screenshot and upload that to Moodle.  Also, let Jane know what you think of the tool. Is it something that LBCC should implement?
 \item Viewing: \cite{decipher}
 \item Supplemental viewing: \cite{gen-ed}
\end{itemize}


\subsubsection{Agenda}
\begin{itemize}
 \item Watch \cite{lasers} and discuss its relevance to various fields
 \item Fill out a schedule for the remaining weeks of the term
 \item Share out any experiences you've had with student services this term
 \item Research help!
\end{itemize}




\subsection{Financial literacy}
\subsubsection{Due}
\begin{itemize}
 \item Assignment: Use the table of contents in \cite{everyday} to identify a gap in your financial knowledge.  Read the relevant chapter, write a response of at least 200 words, and upload it to Moodle.
 \item Reading: \cite{finra}
 \item Supplemental reading: \cite{smart}
\end{itemize}

\subsubsection{Agenda}
\begin{itemize}
\item Financial aid discussion
\item 
\end{itemize}



\subsection{Barriers to college success}
\subsubsection{Due}
\begin{itemize}
 \item Assignment: Identify one passage from the reading that you had a strong reaction to (either negative or positive) and one passage that you didn't understand.  We will use these to kick off our class discussion.
 \item Reading: \cite{guillory2008s}
 \item Supplemental reading: \cite{rosenberg}
\end{itemize}

\subsubsection{Agenda}
\begin{itemize}
\item Discuss \cite{guillory2008s}.
\end{itemize}


\subsection{Critical thinking, evaluating what you read}
\subsubsection{Due}
\begin{itemize}
 \item Assignment: Find an academic journal article and a viewpoint from the \htmladdnormallink{Opposing Viewpoints page on the DREAM Act}{http://ic.galegroup.com.ezproxy.libweb.linnbenton.edu:2048/ic/ovic/PageFinderPortletPage/PageFinderPortletWindow?u=lbcc&disableHighlighting=false&p=OVIC&prodId=&action=1&activityType=SelectedSearch&javax.portlet.action=portalRedirectAction&showDisambiguation=true&catId=GALE\%7CHPZVWQ261035391}.  Use the criteria listed in \cite{evaluation} as you read, then fill out the Moodle worksheet about your article.
 \item Reading: \cite{evaluation}
 \item Supplemental reading: \cite{incorporating,religion}
\end{itemize}

\subsubsection{Agenda}
\begin{itemize}
\item ``acknowledg[ing] our partial and positioned perspectives.'' \cite{finn2003just}
\item Key terms: positionality, epistemology, claim, evidence, hypothesis, theory, methodology, meta-analysis
\item What is a claim?  Case study: text mining.
\item Identify positionality, epistemology, claims, evidence, hypothesis, and methodology in articles from various disciplines
\end{itemize}

\subsection{Participating in scholarly conversations, part 1}
\subsubsection{Due}
\begin{itemize}
 \item Assignment: Uncover a novel fact about something in your community.  Upload a document with your fact and a citation you feel is appropriate, authoritative, and accessible.  Your instructor will evaluate your submission and may add it to Wikipedia.
 \item Reading: \cite{eval}
 \item Supplemental reading: \cite{training}
\end{itemize}

\subsubsection{Agenda}
\begin{itemize}
\item Wikipedia
\item Citations
\item Identifying gaps in an article
\item Primary, secondary, and tertiary sources
\end{itemize}

\subsection{Participating in scholarly conversations, part 2}
\subsubsection{Due}
\begin{itemize}
 \item Assignment: Uncover a novel fact about something in your community.  Add your fact and a citation you feel is appropriate, authoritative, and accessible to a Wikipedia Talk Page.  Your instructor will evaluate your submission and may add it to Wikipedia proper.
 \item Reading: \cite{eryk}
 \item Supplemental reading: \cite{doyle}
\end{itemize}

\subsubsection{Agenda}
\begin{itemize}
\item Review your assignments
\item What to do with feedback from your instructors
\item Discussion: plans for Winter Term
\end{itemize}



\newpage
\renewcommand{\bibname}{Reading list}
\bibliography{dg-reading-list}{}
\bibliographystyle{apalike}


\end{document}