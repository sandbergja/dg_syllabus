\documentclass[12pt,article,oneside]{memoir}
\usepackage{comment}
\usepackage{hyperref}
\usepackage{cite}
\usepackage{html}
\author{Jane Sandberg\thanks{You can email me at \htmladdnormallink{sandbej at linnbenton dot edu}{mailto:sandbej@linnbenton.edu}, call me at \htmladdnormallink{(541) 917 4655}	{tel:5419174655}, or stop by my office hours.}}
\title{HD120: Destination Graduation\thanks{9a Fridays in IA-224}}
\begin{document}
\renewcommand{\labelitemi}{$\triangleright$}
\setcounter{secnumdepth}{0}
\tightlists


\maketitle


\begin{quotation}
Discovery consists of looking at the same thing as everyone else and thinking something different.

\sourceatright{attributed to \htmladdnormallink{Albert Szent-Gy\"{o}rgyi}{ http://search.credoreference.com.ezproxy.libweb.linnbenton.edu:2048/content/topic/szent_gyo\%CC\%88rgyi_albert_1893_1986}}
\end{quotation}

\begin{quotation}
In the classroom, I share as much as possible the need for critical thinkers to engage multiple locations, to address diverse standpoints, to allow us to gather knowledge fully and inclusively.  Sometimes, I tell students, it is like a recipe.  I tell them to imagine we are baking bread that needs flour.  And we have all the other ingredients but no flour.  Suddenly, the flour becomes most important even though it alone will not do.  This is a way to think about [lived] experience in the classroom.

\sourceatright{\htmladdnormallink{bell hooks}{http://search.credoreference.com.ezproxy.libweb.linnbenton.edu:2048/content/topic/hooks_bell_1952}, \emph{Teaching to Transgress}}
\end{quotation}

\begin{htmlonly}
\tableofcontents
\end{htmlonly}
\newpage
\section{Office hours}

\subsection{Mondays}
10:30am-12pm in WH-143 (my office in the Albany campus library)

\section{Required course materials}
\textbf{I do not require any textbooks for this course; please don't buy any.}  All the readings and activities you will need are available as links in this document and Moodle.

\section{Course overview}
Destination Graduation is a ten-hour, one-credit course designed to:
\begin{enumerate}
 \item Help students make a smooth academic and social transition to college life.
 \item Develop students' ability to use tools, information, and resources to be successful at LBCC.
 \item Help students establish a long-term academic advising relationship with a designated academic advisor, most frequently a faculty member.
\end{enumerate}

\section{Learning outcomes}
Upon successful completion of Destination Graduation (DG), you will be able to:
\begin{enumerate}
 \item Identify responsibilities and characteristics of successful students and the barriers to college success.
 \item Apply introductory critical thinking skills.
 \item Navigate important electronic educational resources.
 \item Develop a specific Education Plan related to your career/educational goal.
 \item Locate and know how to access support services and educational resources.
 \item Establish relationships with peers, LBCC faculty, staff, and an advising relationship with a designated academic advisor.
\end{enumerate}

I have an additional learning outcome:
\begin{enumerate}
  \item Develop strategies to engage with the professional/academic writing style of your discipline.
\end{enumerate}

\section{Expectations}
\begin{enumerate}
 \item Attend all class sessions
 \item Participate in discussions (if this is an intimidating context for you, talk with me and we can work on a strategy to get you involved)
 \item Complete all assignments
 \item Come to class prepared
\end{enumerate}


\section{Grading}

In order to pass this course, you must:

\begin{enumerate}
 \item Attend all classes.  I allow one missed class -- no questions asked -- but any other absence \emph{must} be discussed beforehand, unless extenuating circumstances exist.
 \item Complete all assignments in a timely manner.
 \item Complete an Education Plan and get it signed by their advisor.
 \item Meet with their advisor.
 \item Attend one event on campus that is approved by your instructor, and mention it to your colleagues at the beginning of the next class.  LBCC is a vibrant scholarly community, and I'd like you to take advantage of some of the excellent learning opportunities these events provide.  I will be posting possible events to Moodle, but feel free to suggest others to me.
\end{enumerate}

  
\subsection{File formats}
You will be creating most of your assignments in some text editor and submitting them via Moodle.  I am very flexible about which file format you use. In fact, many of the assignments are quite short, so I encourage you to use them to experiment with tools that you are unfamiliar with.  Here are some ideas:
\begin{itemize}
 \item Typical word processor formats, such as those used by Microsoft Word and OpenOffice/LibreOffice
 \item PDF formats
 \item HTML, \LaTeX, LyX,  Docbook, or other typical markup languages
 \item Submit a link to a Google doc, YouTube video, blog post, or other online document you've created that meets the stated criteria 
\end{itemize}



However, there are a few formats that are a total pain to open -- particularly documents created using Apple's \emph{Pages} product.  Please don't submit Pages documents. If you would like to submit something in an offbeat file format not listed here, check with me first to make sure that I can access it with my tools.



\section{Unique features of this DG section}
Like your colleagues in other sections of DG, you will be learning about campus resources and planning for a successful education here at LBCC.  However, a key part of your work at LBCC will be participation in scholarly conversations.  To prepare you for this responsibility, we will spend time thinking about research methodologies, documentation of sources, claims and evidence, novel concepts, and knowledge gaps.  During the last part of this term, we will use these discussions to inform lasting contributions to a frequently-used knowledge base, Wikipedia.



\section{Nondiscrimination and non-harassment statement}

Linn-Benton Community College is committed to providing an atmosphere that  encourages individuals to realize their potential. We embrace diversity and inclusion of all persons. The college prohibits unlawful discrimination based on race, color, religion, ethnicity, use of native language, national origin, sex, sexual orientation, marital status, disability, veteran status, or age in any area, activity or operation of the college. In addition, the college complies with related federal, state, and local laws (Civil Rights, Disability \& Rehabilitation Acts, Veterans Acts).

\section{Disability services and emergency planning statement}

Students who may need accommodations or special tools due to a disability or disabilities should contact the Office of Disability Services (ODS) at 541-917-4789 or Red Cedar Hall 101/103 or access information on the LBCC website (\url{http://linnbenton.edu/disability-services}.) Students who have medical information which the instructor should know, or who need special arrangements in an emergency, should notify their instructor and Public Safety (Red Cedar Hall).

\section{Know your rights (and responsibilities)}

LBCC students have rights: the right to free speech, the right to assemble, the right of a free press, etc. LBCC students also have responsibilities to their community: the responsibility to participate and engage in class, the responsibility to advocate for their needs (ask for help), the responsibility to support a respectful teaching and learning environment, the responsibility to treat all persons with respect, the responsibility to be truthful and honest in all work and communications, and the responsibility to follow staff directions, local, state, and federal laws. Rights and responsibilities balance together to create the best learning environment. For example, while you have free speech in the caf\'{e} or courtyard, in class the instructor decides whose turn it is to talk and what the topics for conversation will be. Students are free to believe what they believe, but instructors may require students to learn and recite concepts, principles, or theories for a class even if the student does not believe those concepts. You play a role in creating a positive community at LBCC. 

Please review your rights and responsibilities at this link: \url{http://linnbenton.edu/go/studentrights}.

If you believe a student is violating your rights, ask to be treated with respect. If that does not cure the situation, report to Associate Dean Dr. Lynne Cox, Takena 107.

If you believe a faculty member or LBCC employee is violating your rights, please report to Human Resources, Scott Rolen, CC-108.

\newpage

\section{Schedule}

All readings and assignments listed are due at the start of the relevant class period.

\subsection{January 8: Introduction to the course and important computer systems}

\subsubsection{Due}
No assignments due

\subsubsection{Agenda}
\begin{itemize}
 \item Introductions: names, potential majors, your motto
 \item Plans for the course
 \item Syllabus
 \item Email, Moodle, WebRunner
 \item Office software options: Office 365, LibreOffice, Google Drive, \LaTeX
\end{itemize}


\subsection{January 15: Resources for student success}
\subsubsection{Due}
\begin{itemize}
 \item Assignment 1: Visit one of the student services sites and have a 15-30 minute conversation with somebody who works there.  Share your findings in class.
 \item Assignment 2: Add a profile picture to your Moodle account.
 \item Viewing: \cite{tannahill} 
\end{itemize}
\subsubsection{Agenda}
\begin{itemize}
\item Share your findings
\item Take survey about your classes
\end{itemize} 








\subsection{January 22: Participating in a scholarly community}
\subsubsection{Due}
\begin{itemize}
 \item Assignment: Write at least 200 words about where you see yourself in 5 years.  Include career goals or any other goals that are important to you.
 \item Reading: \cite{wellsley}
\end{itemize}


\subsubsection{Agenda}
\begin{itemize}
 \item Library and its scholarly research tools
 \item Diversity Achievement Center
 \item Email and text etiquette / strategize for emailing your advisor
 \item Getting started on career skills assignment
\end{itemize}



\subsection{January 29: Preparing for a career}
\subsubsection{Due}
\begin{itemize}
 \item Assignment: Use LBCC or OSU library resources to identify a scholarly article (preferably peer-reviewed) that discusses the skills necessary for a field of study that interests you.  Read the article, and create a document that describes two of these skills, why the author(s) felt they were important, how you can develop them during your time at school. Also include a complete citation for the article you read.  Upload the document to Moodle.
 \item Reading: \cite{worksheets}
 \item Supplemental reading: \cite{crosswalk,ooh}
\end{itemize}

\subsubsection{Agenda}
\begin{itemize}
 \item Explore Occupational Outlook Handbook and QualityInfo
 \item Discuss the degrees, programs, and certificates offered by LBCC
\item Talk about education plans and get started on one
 \item Share out any experiences you've had with student services this term
\end{itemize}


\subsection{February 5: Critical thinking, grit}
\subsubsection{Due}
\begin{itemize}
 \item Assignment: Prepare a draft educational plan.  Make an appointment with your advisor to develop it further.
 \item Viewing: \cite{duckworth}
 \item Reading: \cite{sultan}
\end{itemize}


\subsubsection{Agenda}
\begin{itemize}
\item Special guest: Chareane Wimbley-Gouveia
\item Key terms: positionality, claim, evidence, hypothesis, theory, methodology
\item Talk about grit
\end{itemize}




\subsection{February 12: Financial literacy, time management}
\subsubsection{Due}
\begin{itemize}
 \item Assignment: Use the table of contents in \cite{everyday} to identify a gap in your financial knowledge.  Read the relevant chapter and create a response using a technology you don't typically use.  Upload it to Moodle.
 \item Supplemental reading: \cite{finra, smart}
 \item Supplemental viewing: \cite{credits}
\end{itemize}

\subsubsection{Agenda}
\begin{itemize}
\item Special guest: Rob Priewe
\item Balancing your commitments
\item Financial aid discussion
\item Satisfactory academic progress
\item Fill out a schedule for the remaining weeks of the term
\end{itemize}



\subsection{February 19: Navigating readings}
\subsubsection{Due}
\begin{itemize}
 \item Assignment 1: Go through the reading and fill out the worksheet I've provided.
 \item Assignment 2: Create a Wikipedia account and add a statement about your interests and areas of expertise to your User page.
 \item Reading: \cite{guillory2008s}
 \item Supplemental reading: \cite{rosenberg}
\end{itemize}

\subsubsection{Agenda}
\begin{itemize}
\item Discuss \cite{guillory2008s}.
\end{itemize}


\subsection{February 26: Barriers to college success}
\begin{itemize}
 \item Assignment: Identify one passage from the reading that you had a strong reaction to (either negative or positive) and one passage that you didn't understand.  We will use these to kick off our class discussion.
 \item Reading: \cite{guillory2008s}
\end{itemize}

\subsubsection{Agenda}
\begin{itemize}
\item Discuss \cite{guillory2008s}.
\end{itemize}



\subsection{March 4: Participating in scholarly conversations, part 1}
\subsubsection{Due}
\begin{itemize}
 \item Assignment: Identify a short Wikipedia article about something in your community.  Find some missing information and add a comment to the talk page.
 \item Assignment: Complete your academic plan.
 \item Reading: \cite{eval}
 \item Supplemental reading: \cite{training}
\end{itemize}

\subsubsection{Agenda}
\begin{itemize}
\item Wikipedia
\item Citations
\item Primary, secondary, and tertiary sources
\item Getting ready for finals
\end{itemize}


\subsection{March 11: Participating in scholarly conversations, part 2}
\subsubsection{Due}
\begin{itemize}
 \item Assignment: Uncover a novel fact that is missing from the article you identified last week.  Add your fact and a citation for a secondary source to a Wikipedia Page.  Your source must be appropriate, authoritative, and accessible.
 \item Reading: \cite{eryk}
 \item Supplemental reading: \cite{doyle}
\end{itemize}

\subsubsection{Agenda}
\begin{itemize}
\item Review your assignments
\item What to do with feedback from your instructors
\item Discussion: plans for next term
\end{itemize}


\newpage
\renewcommand{\bibname}{Reading list}
\bibliography{dg-reading-list}{}
\bibliographystyle{apalike}


\end{document}
